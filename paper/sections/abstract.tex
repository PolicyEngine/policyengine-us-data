\section*{Abstract}

We combine the demographic detail of the Current Population Survey (CPS) with the tax precision of the IRS Public Use File (PUF) to create an enhanced microsimulation dataset. Our method uses quantile regression forests to transfer income and tax variables from the PUF to demographically-similar CPS households, followed by a dropout-regularized gradient descent procedure that reweights households to match administrative targets. The enhanced dataset reduces discrepancies in key tax components by 40\% compared to the baseline CPS while preserving demographic relationships and program participation patterns. Validation against IRS Statistics of Income shows the enhanced data captures capital gains within 12\% of administrative totals (vs. 45\% baseline error), business income within 8\% (vs. 38\%), and dividend income within 7\% (vs. 32\%). The dataset matches state-level EITC claims within 5\% for 45 states and maintains the CPS's high accuracy for poverty estimation and program participation analysis. We release both the enhanced dataset and our open-source enhancement procedure to support transparent policy analysis.