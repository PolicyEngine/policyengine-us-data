\section{Data Sources}

Our methodology combines two primary data sources with calibration targets from six administrative sources.

\subsection{Primary Data Sources}

\subsubsection{Current Population Survey (CPS)}

The Current Population Survey Annual Social and Economic Supplement (ASEC) serves as our base dataset. Conducted jointly by the Census Bureau and Bureau of Labor Statistics, the CPS ASEC surveys approximately 75,000 households annually.

Key features:
\begin{itemize}
\item Representative sample of US households
\item Detailed demographic information
\item Family and household relationships
\item Geographic identifiers including state
\item Program participation questions
\item Self-reported income by source
\end{itemize}

Limitations:
\begin{itemize}
\item Income underreporting, especially at high incomes
\item Limited tax detail
\item No information on itemized deductions
\item Topcoding of high values
\end{itemize}

\subsubsection{IRS Public Use File (PUF)}

The IRS Statistics of Income Public Use File contains detailed tax return information from a stratified sample of individual income tax returns. The most recent PUF available is from tax year 2015, containing approximately 230,000 returns.

Key features:
\begin{itemize}
\item Accurate income reporting from tax returns
\item Detailed breakdown of income sources
\item Complete deduction information
\item Tax credits and payments
\item Sampling weights to represent all filers
\end{itemize}

Limitations:
\begin{itemize}
\item No demographic information beyond filing status
\item No state identifiers
\item Excludes non-filers
\item Significant time lag (2015 data)
\item No household structure
\end{itemize}

\subsection{Additional Data Sources for Imputation}

Beyond the PUF, we incorporate data from three additional surveys to impute specific variables missing from the CPS:

\subsubsection{Survey of Income and Program Participation (SIPP)}

The SIPP provides detailed income and program participation data. We use SIPP to impute:
\begin{itemize}
\item \textbf{Tip income}: Using a Quantile Regression Forest model trained on SIPP data, we impute tip income based on employment income, age, and household composition
\end{itemize}

\subsubsection{Survey of Consumer Finances (SCF)}

The SCF provides comprehensive wealth and debt information. We use SCF to impute:
\begin{itemize}
\item \textbf{Auto loan balances}: Matched based on household demographics and income
\item \textbf{Interest on auto loans}: Calculated from imputed balances
\item \textbf{Net worth components}: Various wealth measures not available in CPS
\end{itemize}

The SCF imputation uses their reference person definition (male in mixed-sex couples or older person in same-sex couples) to ensure proper matching.

\subsubsection{American Community Survey (ACS)}

The ACS provides detailed housing and geographic data. We use ACS to impute:
\begin{itemize}
\item \textbf{Property taxes}: For homeowners, imputed based on state, household income, and demographics
\item \textbf{Rent values}: For specific tenure types where CPS data is incomplete
\item \textbf{Housing characteristics}: Additional housing-related variables
\end{itemize}

These imputations use Quantile Regression Forests to preserve distributional characteristics while accounting for household heterogeneity.

\subsection{Calibration Data Sources}

We calibrate the enhanced dataset to over 7,000 targets from six authoritative sources:

\subsubsection{IRS Statistics of Income (SOI)}

Annual tabulations from tax returns provide income distributions by:
\begin{itemize}
\item Adjusted Gross Income (AGI) bracket
\item Filing status
\item Income type
\end{itemize}

We use SOI Table 1.4 which cross-tabulates income components by AGI ranges, creating over 5,300 distinct targets.

\subsubsection{Census Population Projections}

National and state-level demographic targets from:
\begin{itemize}
\item Single-year-of-age populations (ages 0-85)
\item State total populations
\item State populations under age 5
\end{itemize}

\subsubsection{Congressional Budget Office (CBO)}

Program participation and revenue projections:
\begin{itemize}
\item SNAP (Supplemental Nutrition Assistance Program)
\item Social Security benefits
\item Supplemental Security Income (SSI)
\item Unemployment compensation
\item Individual income tax revenue
\end{itemize}

\subsubsection{Joint Committee on Taxation (JCT)}

Tax expenditure estimates for major deductions:
\begin{itemize}
\item State and local tax deduction: \$21.2 billion
\item Charitable contribution deduction: \$65.3 billion
\item Mortgage interest deduction: \$24.8 billion
\item Medical expense deduction: \$11.4 billion
\end{itemize}

\subsubsection{Treasury Department}

Additional program totals:
\begin{itemize}
\item Earned Income Tax Credit by number of children
\item Total EITC expenditure
\end{itemize}

\subsubsection{Healthcare Spending Data}

Age-stratified medical expenditures:
\begin{itemize}
\item Health insurance premiums (excluding Medicare Part B)
\item Medicare Part B premiums
\item Other medical expenses
\item Over-the-counter health expenses
\end{itemize}

\subsection{Data Preparation}

\subsubsection{CPS Processing}

We use the CPS ASEC from survey year 2024 (covering calendar year 2023 income). The Census Bureau provides:
\begin{itemize}
\item Person-level records with demographics
\item Hierarchical identifiers linking persons to families and households
\item Initial survey weights
\end{itemize}

\subsubsection{PUF Processing}

The 2015 PUF requires several adjustments:
\begin{itemize}
\item Dollar amounts uprated using SOI growth factors by income type
\item Records filtered to remove those with insufficient data
\item Weights normalized to represent the filing population
\end{itemize}

\subsubsection{Target Preparation}

Administrative targets are collected for the appropriate year:
\begin{itemize}
\item Most targets use 2024 projections
\item Historical data uprated using official growth rates
\item State-level targets adjusted for population changes
\end{itemize}

The combination of these data sources enables us to create a dataset that maintains the CPS's demographic richness while achieving tax reporting accuracy comparable to administrative data.