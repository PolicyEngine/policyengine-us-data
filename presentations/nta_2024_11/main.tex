\documentclass{beamer}
\usetheme{Madrid}
\usepackage{policyengine}
\usepackage{graphicx}
\usepackage{booktabs}

\title{Enhancing the CPS with Administrative Tax Data}
\subtitle{Machine Learning Meets Microsimulation}
\author[Woodruff \& Ghenis]{Nikhil Woodruff \& Max Ghenis}
\institute{PolicyEngine}
\date{National Tax Association Annual Conference\\November 14, 2024}

\begin{document}

\begin{frame}
    \titlepage
\end{frame}

\begin{frame}{The Challenge}
    \begin{itemize}
        \item Current Population Survey (CPS)
        \begin{itemize}
            \item Rich demographics, state ID, household structure
            \item Underreports income, especially at top
            \item Limited tax information
            \item Self-reported program participation
        \end{itemize}
        \item IRS Public Use File (PUF)
        \begin{itemize}
            \item Accurate tax records from administrative data
            \item No demographics beyond age/sex
            \item No state ID
            \item Strict confidentiality rules
        \end{itemize}
        \item Need both for comprehensive policy analysis
    \end{itemize}
\end{frame}

\begin{frame}{Key Innovation: Open Dataset with PUF Quality}
    \begin{itemize}
        \item First openly available dataset integrating CPS and PUF
        \item No confidentiality restrictions
        \item Preserves demographic detail while matching tax data
        \item Powers PolicyEngine microsimulation platform
        \item Enables:
        \begin{itemize}
            \item Analysis by race, education, disability status
            \item Program interactions
            \item Transparent, reproducible research
        \end{itemize}
    \end{itemize}
\end{frame}

\begin{frame}{Our Approach}
    \begin{itemize}
        \item Generate synthetic tax variables from PUF using:
        \begin{itemize}
            \item Quantile regression forests to learn distributions
            \item Match to CPS demographic variables
        \end{itemize}
        \pause
        \item Stack synthetic records with original CPS
        \pause
        \item Calculate taxes and benefits via microsimulation
        \pause
        \item Optimize weights against 570 targets:
        \begin{itemize}
            \item IRS Statistics of Income income bins
            \item Program participation totals
            \item Single-year age population counts
        \end{itemize}
    \end{itemize}
\end{frame}

\begin{frame}{Enhanced CPS Pipeline}
    \begin{figure}
        \centering
        \includegraphics[width=\textwidth]{../../paper/figures/data_flow.png}
        \caption{Overview of dataset enhancement process}
    \end{figure}
\end{frame}

\begin{frame}{Machine Learning Methods}
    \begin{itemize}
        \item Quantile Regression Forests (QRF) for:
        \begin{itemize}
            \item Imputing tax variables from PUF
            \item Predicting housing costs from ACS
            \item Estimating prior year earnings
        \end{itemize}
        \item Benefits of QRF approach:
        \begin{itemize}
            \item Captures full conditional distributions
            \item Handles non-linear relationships
            \item Preserves correlations between variables
            \item Outperforms traditional statistical matching
        \end{itemize}
    \end{itemize}
\end{frame}

\begin{frame}{Machine Learning Methods}
    \begin{figure}
        \centering
        \includegraphics[width=\textwidth]{../../paper/figures/quantile_loss.png}
        \caption{Average quantile loss by method, predicting net worth from covariates in SCF}
    \end{figure}
\end{frame}

\begin{frame}{Weight Optimization}
    \begin{itemize}
        \item Novel dropout-regularized gradient descent
        \item Optimizes against 570 targets including:
        \begin{itemize}
            \item IRS Statistics of Income by income bins
            \item Program participation totals
            \item Single-year age population counts
            \item State-level aggregates
        \end{itemize}
        \item Prevents overfitting to any single target
        \item Handles sparse subgroups effectively
    \end{itemize}
\end{frame}

\begin{frame}{Weight Optimization Mathematics}
    \begin{itemize}
        \item Minimize relative squared error:
        \vspace{0.5cm}
        \[ L(w) = \text{mean}\left(\left(\frac{w^T M + 1}{t + 1} - 1\right)^2\right) \]
        \vspace{0.5cm}
        where:
        \begin{itemize}
            \item $w$ are log-transformed household weights
            \item $M$ is matrix of household characteristics
            \item $t$ are target values from administrative data
        \end{itemize}
    \end{itemize}
    \begin{itemize}
        \item Implementation:
        \begin{itemize}
            \item PyTorch gradient descent with Adam optimizer
            \item 5\% dropout rate for regularization
            \item 5,000 iterations or convergence
        \end{itemize}
    \end{itemize}
\end{frame}

\begin{frame}{Validation: Example Targets}
    \begin{table}
        \small
        \centering
        \begin{tabular}{l@{\hspace{1em}}r@{\hspace{1em}}r@{\hspace{1em}}r@{\hspace{1em}}r}
    \toprule
    & Admin & CPS & PUF & ECPS \\
    \midrule
    Qual Div & \$314b & \$103b (-67\%) & \$263b (-16\%) & \$322b (+3\%) \\[0.3ex]
    Infants & 3.6m & 2.8m (-23\%) & 17.0m (+367\%) & 4.0m (+11\%) \\[0.3ex]
    AGI 100-200k & 24.2m & 29.5m (+22\%) & 24.3m (+0\%) & 28.3m (+17\%) \\
    \bottomrule
\end{tabular}
    \end{table}
    \begin{itemize}
        \item ECPS is best on qualified dividends and infant population
        \item PUF better on returns AGI 100-200k
        \item 567 other targets!
    \end{itemize}
\end{frame}

\begin{frame}{Validation: Target Calibration}
    \begin{figure}
        \centering
        \includegraphics[width=\textwidth]{../../paper/figures/ecps_vs_cps_puf.png}
        \caption{Error change from ECPS to better of CPS and PUF}
    \end{figure}
    \begin{itemize}
        \item ECPS outperforms CPS on 63\% of targets
        \item ECPS outperforms PUF on 71\% of targets
    \end{itemize}
\end{frame}

\begin{frame}{Validation: Income Distribution}
    \begin{table}
        \centering
        \begin{table}[h]
    \centering
    \caption{Key tax unit-level distributional metrics}
    \label{tab:tax_unit_metrics}
    \begin{tabular}{lrrr}
    \toprule
    Metric & CPS & Enhanced CPS & PUF \\
    \midrule
    Gini coefficient & 0.495 & 0.572 & 0.570 \\
    Top 10\% share & 0.361 & 0.425 & 0.410 \\
    Top 1\% share & 0.085 & 0.154 & 0.150 \\
    \bottomrule
    \end{tabular}
\end{table}

    \end{table}
    \begin{itemize}
        \item CPS inequality measures 12-45\% lower than PUF
        \item ECPS inequality within 4\% of PUF
        \item Inequality measured as income after taxes and transfers
    \end{itemize}
\end{frame}

\begin{frame}{Validation: Policy Analysis}
    \begin{itemize}
        \item Test case: Biden's proposed top rate increase
        \item Would raise rate from 37\% to 39.6\% above \$400k
    \end{itemize}
    \begin{table}
        \centering
        \begin{table}[h]
    \centering
    \caption{Projected revenue from top rate increase, 2025}
    \label{tab:top_rate_reform}
    \begin{tabular}{lr}
    \toprule
    Source & Revenue (billions) \\
    \midrule
    Treasury & \$75.4 \\
    Enhanced CPS & \$75.7 \\
    Baseline CPS & \$28.7 \\
    \bottomrule
    \end{tabular}
\end{table}

    \end{table}
\end{frame}

\begin{frame}{The PolicyEngine Platform}
    \begin{figure}
        \centering
        \includegraphics[width=\textwidth]{../../paper/figures/policyengine_policy.png}
        \caption{PolicyEngine's policy editor interface}
    \end{figure}
\end{frame}

\begin{frame}{Interactive Results}
    \begin{figure}
        \centering
        \includegraphics[width=\textwidth]{../../paper/figures/policyengine_results.png}
        \caption{Example distributional analysis from PolicyEngine}
    \end{figure}
\end{frame}

\begin{frame}{Unique Demographic Analysis}
    \begin{itemize}
        \item Direct race/ethnicity analysis without imputation
        \item Other models must use complex methods:
        \begin{itemize}
            \item CBO: Statistical matching with Census data
            \item Tax Policy Center: Multiple copies with reweighting
            \item ITEP: Probability assignment based on characteristics
        \end{itemize}
        \item Our approach:
        \begin{itemize}
            \item Uses observed demographics from CPS
            \item Individual-level rather than tax unit only
            \item Enables analysis of intersectional effects
            \item Extends to disability, education, etc.
        \end{itemize}
    \end{itemize}
\end{frame}

\begin{frame}{Open Science}
    \begin{itemize}
        \item Full codebase on GitHub
        \item Automatic validation dashboard
        \item Python package for programmatic access
        \item Web interface at policyengine.org
        \item Growing research applications:
        \begin{itemize}
            \item Academic studies
            \item Think tank analysis
            \item Government agency use
        \end{itemize}
    \end{itemize}
\end{frame}

\begin{frame}{Next Steps}
    \begin{itemize}
        \item Geographic extensions:
        \begin{itemize}
            \item Congressional district weights
            \item State-specific calibration
            \item County-level synthetic data
        \end{itemize}
        \item Methodological improvements:
        \begin{itemize}
            \item Time series validation
            \item Uncertainty quantification
            \item Alternative ML architectures
        \end{itemize}
        \item International applications (UK version live)
    \end{itemize}
\end{frame}

\begin{frame}{Thank You}
    \begin{itemize}
        \item Paper: github.com/PolicyEngine/policyengine-us-data/paper
        \item Code: github.com/PolicyEngine/policyengine-us-data
        \item Web app: policyengine.org
        \item Contact: max@policyengine.org
    \end{itemize}
\end{frame}

\end{document}